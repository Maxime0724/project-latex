\documentclass[a4paper,french,10pt]{report}
\usepackage[french]{babel}
\usepackage[utf8]{inputenc}
\usepackage{graphicx}
\usepackage{float}
\usepackage{amsmath}
\usepackage{amssymb}
\usepackage[T1]{fontenc}

\begin{document}
	\title{Rapport TP \\ \LaTeX}
	\author{GUAN Jingyuan \\ nf92p002}
	\date{Le \today}
	\maketitle
	\tableofcontents
	\listoffigures
	\listoftables
	
\begin{abstract}
    \qquad Voilà mon premier rapport de NF92. J'ai presque fini le rapport pendant 3 semaines. J'ai réalisé les fabriquation de tableaux et l'insertion d'images. Dans le chapitre \ref{chapitre1}, j'ai mis un tableau dans le section \ref{section1} et 2 images dans le section \ref{section2}. Il y a plusieurs équations mathématiques dans le chapitre \ref{chapitre2}, surtout la section \ref{section3} et la section \ref{section5}.
    
\end{abstract}
	
	\begin{center}
		\chapter{Début}
		\label{chapitre1}
	\end{center}
	\section{Les caractères spéciaux}
	\label{section1}
	\begin{table} [h]\begin{center}
		\begin{tabular}{|c|c|}
  			\hline
  			Caractères spéciaux & commandes \\
  			\hline
			\hline
  			\$ & $\backslash$\$ \\
  			\hline
			\& & $\backslash$\& \\
			\hline
			\% & $\backslash$\% \\
			\hline
			\# & $\backslash$\# \\
			\hline
			\{ & $\backslash$\{ \\
			\hline
			\} & $\backslash$\} \\
			\hline
			\_ & $\backslash$\_ \\
			\hline
			\LaTeX & $\backslash$LaTeX \\
			\hline
		\end{tabular}
	\caption{Table des caractèrers spéciaux}	
	\label{table1}	
	\end{center} \end{table} 
	Regardez la table \label{table1}, j'ai utilisé \verb|$backslach$| pour exprimer $\backslash$, et j'ai exprimé les codes entre \$ avec \verb|\verb| et tapé les codes entre $|$. J'ai nommé le tableau par "table1".
	
	\section{Les images}
	\label{section2}
	\begin{figure}[h]
		\centering
		\includegraphics[scale=0.3]{fedora.eps}
		\caption{Un logo fedora core}
		\label{figure1}
	\end{figure}	
	
	\begin{figure}[H]
		\begin{center}
		\includegraphics[scale=0.3, angle=20]{fedora.eps}
		\caption{Un logo fedora core avec un angle de 20$^o$}
		\label{figure2}
		\end{center}
	\end{figure}
	Pour que la figure \ref{figure2}  ait un angle, j'ai écrit "angle=20". J'ai ajouté le package[float] pour placer la figure \ref{figure2} au dessus de la page.

	\begin{center}
			\chapter{Travail réalisé}
			\label{chapitre2}
	\end{center}
	\section{Références aux formules}
	\label{section3}
	Le déterminant d'une matrice $3 \times 3$ est : 
	\begin{center}
		$|A|=\left|
		\begin{array}{lrc}
		\ a & b & c \\
		\ d & e & f \\
		\ g & h & i \\
		\end{array}
		\right|
		=a \times \left|
		\begin{array}{cc}
		\ e & f \\
		\ h & i \\
		\end{array} \right|
		-b \times \left|
		\begin{array}{cc}
		\ d & f \\
		\ g & i \\
		\end{array} \right|
		+c \times \left|
		\begin{array}{cc}
		\ d & e \\
		\ g & h \\
		\end{array} \right|$
	\end{center} \par
	Soit: 
	\begin{equation}
		\label{equation1}
	    \Delta (A) = aei+bfg+cdh-ceg-bdi-afh
	\end{equation}
Si $\Delta(A) = |A| = det(A) = 0$, calculé avec l'équation \ref{equation1}, alors il n'y a pas une unique solution au système matriciel. \\
Ce qui n'a strictement rien à voir avec le travail présenté dans les articles $[1, 3]$.\\
\cite{burke1997automated}


\section{Exemples de graphes}
\label{section4}
\begin{figure}[h]
		\centering
		\includegraphics[scale=0.4]{K5_K3_3_Graph.eps}
		\caption{A gauche la clique $K5$, à droite $K3\_ 3$ un graphe biparti complet}
		\label{figure3}
\end{figure}
Considérons la Figure (\ref{figure3}):cinq couleurs son necessaires pour colorier le graphe $K5$, tandis que deux suffisent pour $K3\_3$. Cependant  un graphe est planaire s’il ne contient parmi ses mineurs aucun des graphes de la Figure (\ref{figure3}).


\section{The original mode}
\label{section5}
\noindent We address the examination timetabling problem proposed in the second International Timetabling
Competition (ITC2007). The reader should refer to \cite{burke1997automated} for a detailed overview of examination
timetabling.\par
We present in the sequel the original model
proposed by \cite{springerlink:10.1007/s10479-011-0997-x}. As stated by the authors, the aim was to give a clear model. We invite the reader to refer to the original paper for comprehensive details.\par
In \cite{springerlink:10.1007/s10479-011-0997-x}, the authors introduced the conflict graph $G(E, A_C )$, where $E$ is the set of exams
and an edge $[i, j] \in AC$ if there is at least one student enrolled in exams $i$ and $j$. An edge
$[i, j]$ is weighted by $w^{C}_{ij}$, the number of students taking the two exams. The core problem
is to find a graph coloring. Note that \cite{sabar2012graph} derived a hyper-heuristic based on graph coloring
constructive ordering heuristics to select exams to be scheduled. $P$, $R$ and $S$ denote the
sets of periods, rooms and students respectively. For the sake of compactness, the objective
function and the hard constraints of the model have been rewritten as:\par
Minimize:
	\begin{equation}
	 C^{2D}
	 \label{equation2}
	\end{equation} \\
Subject to \par
	\begin{equation}
	 \forall i\in E\quad\sum_{p\in P}\sum_{r\in R}X^{PR}_{ipr} \leq 1
	\label{equation3}
	\end{equation}
	
	\begin{equation}
	    \forall p\in P\quad\forall r\in R\quad\sum_{i\in E}s^E_iX^{PR}_{ipr} \leq s^R_r
	\label{equation4}
	\end{equation}
	
    \begin{equation}
        \forall i\in E \quad\forall p\in P \quad \sum_{j\in N(i)}X^{P}_{jp} + a_{ip}X^P_{ip}\leq a_{ip}
    \label{equation5}
    \end{equation}
    
    \begin{equation}
        C^{2D} = w^{2D}\sum_{[i,j]\in A_C} w^C_{ij}C^{2D}_{ij}
    \label{equation6}
    \end{equation}
    
    \begin{equation}
        \left.
        \begin{aligned}
        \forall [i,j]\in A_C\quad\forall p,q\in P\quad with \quad|p-q|=1 \quad\\
        with\quad y_{pq}=1\quad X^P_{ip}+X^P_{jq}\leq 1+C^{2D}_{ij}\quad     
        \end{aligned}
        \right\}
    \label{equation7}
    \end{equation}\par
    
Equations (\ref{equation3}) ensure that all the exams are allocated once to a unique period and a unique
room. The room capacities are always respected using Equations (\ref{equation4}) in which $s^E_i$
and $s^R_r$ denote the number of students sitting exam $i$ and the seating capacity of room $r$ respectively.\par
Equations (\ref{equation5}) enforce the conflict constraints: at any period, any student will be sitting
at most one exam.\par 
The boolean variables used to count the number of \textbf{Two In a Day} penalties are $C^{2D}_{ij}$,
$C^{2D}_{ij}=1\quad iff$ two exams are allocated in the same day but not back to back. The integer
variable $C^{2D}$ is used to compute the objective function (see Equation (\ref{equation2})).\par 
The \textbf{Two In a Day} term $C^{2D}$ is set by Equations (\ref{equation6}) and (\ref{equation7}), in which the boolean
parameter $y_{pq} = 1\quad iff$ periods $p$ and $q$ are on the same day.\par
\begin{table}[H]
 \centering
     \caption{Characteristics of the Yeditepe datasets}
     \begin{tabular}{c|cccc||cc||c}
      & $n^E$ & $n^S$ & $n^P$ & $n^R$ & $w_{A_C}$ & $n_{w_{A_C}}$ & $t_{\alpha_{ip}}$ \\
      \hline
     yue20011 & 126 & 569 & 18 & 2 & 14 & 78 & 1 \\
     yue20012 & 141 & 581 & 18 & 2 & 17 & 8 & 0 \\
     yue20023 & 38 & 224 & 6 & 1 & 6 & 4 & 1 \\
    \end{tabular}
    \label{table2}
\end{table}
The characteristics of the Yeditepe datasets used for our tests are displayed in Table \ref{table2}.\par

\bibliographystyle{plain}
%\bibliographystyle{abbr}
%\bibliographystyle{alpha}

\bibliography{biblio}
\end{document}

	
